\documentclass[prb,preprint]{revtex4-1} 

\usepackage{amsmath}  
\usepackage{amsfonts} 
\usepackage{graphicx} 
\usepackage{gensymb}

\begin{document}

\title{Measurement of Faraday Rotation and Calculation of the Verdet Constant for a SF-59 Glass Rod}

\author{Frances Yang}
\email{fyang@smith.edu} 

\author{Isabel Lipartito}
\email{iliparti@smith.edu}
\affiliation{Department of Physics, Smith College, Northampton, MA 01063}

\date{\today}

\begin{abstract}
{We observed the Faraday effect, a shift in the polarization of light as it passes through medium subject to a magnetic field. The Verdet constant is relates the change in the magnetic field to a change in the polarization.  Polarized light from a 650 nm laser was sent through a SF-59 glass rod surrounded by a solenoid. The light then passed through an rotatable polarizer and was detected by a photodiode. A lock-in amplifier was used on the photodiode output to reduce noise from background light and DC drift from the photodiode. We calculated the Verdet constant for the rod by varying the magnetic fields and measuring the relative polarization shift and the voltage shift at a particular polarizer angle. From two separate calculations, we found a Verdet constant of $15.1 \pm 1.2 \mathrm{~rad/T} \cdot \textrm{m}$ and $19.88 \pm 0.11 \mathrm{~rad/T} \cdot \textrm{m}$. {\bf compare our values with other groups here
}}
\end{abstract}

\maketitle 
\section{Aims}
{a.  To demonstrate the Faraday effect of the rotation of the plane of polarization of light as it travels through magnetic fields of different magnitudes.

b.  To calculate the Verdet constant relating the change in polarization angle to the change in magnetic field.}
\begin{equation}
\Delta \phi =V_{c} \Delta B L{_{sample}}
\end{equation}

\section{Introduction} 

{The Faraday Effect is a magneto-optical phenomenon in which light of a single wavelength, traveling through certain materials subject to a magnetic field, experiences a shift in the plane of polarization. These materials, called birefringent, have different refractive indices for the right and left circular polarizations of light. This changes the relative phase angle between the two polarizations as the light travels through, thus rotating the overall polarization plane.

As shown in Equation 1, the Verdet Constant ($V_{c}$) relates the change in magnetic field of a medium to the change in polarization angle of the traveling light.  $V_{c}$ is specific to any medium.  There are two ways in which we can calculate $V_{c}$.  By Malus's Law, $I(\theta)$=$I_{0}*\cos^{2}(\theta)$, where $I$ is the intensity of measured light through a polarization filter of angle $\theta$ with respect to the maximal polarization angle.  Similarly, for voltage output, $V(\theta)$=$V_{0}*\cos^{2}(\theta)$.  Once we add into this set-up a magnetic field through which the light can travel, we will observe the voltage output equation to have a phase shift, $\theta$:  $V(\theta)$=$V_{0}*\cos^{2}(\theta+\phi)$.

One way to calculate $V_{c}$ is to measure the phase shift, $\phi$ for multiple magnetic fields:  collecting data for $V(\theta)$ for a full 0-360 degree range of $\theta$ and comparing data for different magnetic fields to a base data set of the same set up, where there is no magnetic field.  $\theta$ can be plotted against $B$- they should have a linear relationship- and by finding the slope of that linear fit, $\frac{\partial \theta}{\partial B}$ will be found and thus $V_{c}$ by Equation 1.}

{Another way to find $V_{c}$ is to notice that $\frac{\partial \theta}{\partial B}= \frac{\partial \theta}{\partial V}*\frac{\partial V}{\partial B}$}

\section{Procedure}
{Polarized light came from a 650 nm laser, driven by a 400 kHZ square wave from a function generator. A 15.2 cm long solenoid with 1400 turns and a DC resistance of 1.6 $\Omega$ provided a magnetic field that could be varied by changing the current passing through the solenoid. Current was generated by a Keithley 2230-30-1 DC Power Supply.  Channel 1 and Channel 2 were connected in parallel to provide sufficient current. The calibration of the solenoid was found by measuring the magnetic field at the center of the solenoid for various currents (0 A to 2.25 A in steps of 0.25 A) using a TEL-Atomic Smart Magnetic Sensor (model SMS 102).   The calibration constant was measured to be 10.6 mT/A $\pm$ .05 mT. Our sample, a 10.2 $\pm$ 0.5 cm SF-59 glass rod, was centered inside the solenoid. The light is passed through a polarizer and detected by a photodiode, connected to a 1 k$\Omega$ resistor.

A lock-in amplifier was used to rectify the signal from the photodiode.  The output of the photodiode was connected to a band pass filter with a center frequency of 400 kHz and a Q of 20, which was fed into a lock-in of gain 5.  The reference signal for the lock-in amplifier was a 400 kHz sine wave of amplitude 1 V peak-to-peak, provided by the same function generator driving the laser, so the phases could be aligned.  The output signal from the lock-in was fed to a low-pass filter-amplifier with a time constant of .1 seconds and the resultant voltage was read by a Keithley 2100 Digital Multimeter.

Voltage readings from the multimeter were recorded onto the computer by LabView program (Keithley DC Incremental Write.vi). Each recorded voltage was an average of 16 measurements. Voltages were recorded for a 360\degree\ rotation of the polarizer at increments of 5\degree. We used this procedure to measure the voltage output for a current of -1 A, 0 A, 1.5 A, 2 A, corresponding to a B Field of -10.6 mT, 0 mT, 15.9 mT, and 21.2 mT respectfully. The negative current was generated by reversing the connection of the two leads of the power supply.

}




\section{Results}
{Data recorded by LabView was fed into a data analysis program called Igor.Pro.  A sinusoidal fit was determined for each separate dataset corresponding to a different current value.  We excluded two points in the curve fit for the B=0 dataset, which corresponded to values we accidentally recorded twice. For all data sets, the values corresponding to angles of 320\degree--330\degree were masked from the curve fits as well, due to appearance of systematic errors. 
\begin{figure}
\includegraphics[width = 6.3in]{0A.pdf}
\caption{\label{nofield}Voltage measurements }

\end{figure}

\begin{figure}[b]
\includegraphics[width =6.3in]{change_in_voltage_for_different_b_fields_at_45_from_max.pdf}
\caption{\label{method2pic} The measured change in photodiode voltage for a change in the magnetic field at a polarizer angle 45\degree\  from the maximum voltage output. A linear fit was used to determine $\Delta V/\Delta B$.}
\end{figure}
}
\section{Analysis}


\section{Discussion}




\section{Conclusion}


\section{References}

\end{document}