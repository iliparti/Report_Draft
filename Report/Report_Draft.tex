\documentclass[prb,preprint]{revtex4-1} 

\usepackage{amsmath}  
\usepackage{amsfonts} 
\usepackage{graphicx} 

\begin{document}

\title{Measurement of Faraday Rotation and \\ Calculation of the Verdet Constant }

\author{Frances Yang}
\email{fyang@smith.edu} 
\altaffiliation[permanent address: ]{101 Main Street, 
  Anytown, USA} 
\affiliation{Department of Physics, Smith College, Northampton, MA 01063}

\author{Isabel Lipartito}
\email{iliparti@smith.edu}
\affiliation{Department of Physics, Smith College, Northampton, MA 01063}

\date{\today}

\begin{abstract}
We aim to calculate the Verdet Constant relating the change in magnetic field through a medium sample to the change in polarization angle of light traveling through the medium.  Light from a (?wavelength?) laser was sent through an adjustable polarizer and then medium sample inside asolenoid capable of producing magnetic fields from 0 to 32 mT and was received by a photodiode.  Output voltage ($V$) magnitude was plotted against polarization angle ($\theta$) for four different $B$ fields.  The Verdet Constant for the medium used was calculated considering the change in phase of $V$ vs. $\theta$ plots with varying magnetic field, resulting in a value of 20.1 $\frac{1}{mT*m}$ ERROR.
\end{abstract}

\maketitle 
\section{Aims}
a.  To demonstrate the Faraday effect of the rotation of the plane of polarization of light as it travels through magnetic fields of different magnitudes.

b.  To calculate the Verdet Constant relating the change in magnetic field to the change in polarization angle.
\begin{equation}
\partial \theta =C_{V}*\partial B*L{_{sample}}
\end{equation}

\section{Introduction} 

{The Faraday Effect is a magneto-optical phenomenon in which light of a single wavelength, traveling through a medium of a certain refractive index subject to a magnetic field, experiences a shift in the plane of polarization.  This has to do with the fact that light has a right-hand circularly polarized component and a left-hand circularly polarized component.  If it is sent through a medium considered to be birefringent (having a different refractive index for light in different polarization orientations), the relative phase angle between the two components will have changed and the overall plane of polarization will have rotated.}

{SEAL SEAL SEAL}


\section{Procedure}



\section{Results}


\section{Analysis}


\section{Discussion}




\section{Conclusion}


\section{References}

\end{document}